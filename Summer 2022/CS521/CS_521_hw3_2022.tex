\documentclass[margin=3mm]{article}
\usepackage[edges]{forest}
\oddsidemargin 0.15in
\textwidth 6.25in
\topmargin-0.85in
\textheight 9.0in
\headsep 0.6in
\begin{document}
\begin{center}
\begin{LARGE}{\bf CS 521: Data Structures and Algorithms 1\\Summer
      2022 (Homework 3)} \end{LARGE}
\end{center}

For an extra 5 credit points submit your answers as a latex compiled with pdf. Please submit the .tex file and the .pdf file.


\begin{enumerate}
\item (25 Pts. )Preorder traversals of binary search trees are quite different: given a preorder traversal of a binary search tree, one can reconstruct the tree uniquely (assuming all the keys are distinct). Design an algorithm that, given an array of n distinct elements, constructs the unique binary search tree whose preorder traversal is given by the array. To get full credit you must show pseudo-code for your algorithm, show it is true using loop invariants, prove it’s running time with one of the three methods we discussed in class).

\item[Ans: ]  The concept of binary search tree and preorder tree traversal is very clear. As pre word discribe it as Root then Left and then Right. As per the pseudo code of preorder traversal we will traverse the tree using $O(n^2)$ time complexity.Because we are going like these root left right recursion. but we can do it in O(n) time if we check the number or key with the root compare it and place it using the binary search tree methods. I was working on leetcode there is problem of BST preorder traversal it is similar to that only I will show it here.\\
The Pseudo code for the Preorder Tree Traversal.\\
procedure preorder(p : pointer to a tree node)\\
    if p != nullptr\\
        Visit the node pointed to by p\\
        preorder(p$->$left)\\
        preorder(p$->$right)\\
    end if\\
end procedure\\

Given preorder traversal of a binary search tree to construct the BST\\
Preorder - 20, 16, 5, 18, 17, 19, 60, 85, 70 (Root,Left,Right)\\
Inorder - 5, 16, 17, 18, 19, 20, 60, 70, 85 ((Left, Root, Right)\\
Root is always the initial component of a preorder traversal. The root is first built. The next step is to determine the first element's index that exceeds the root. Let i be the index. The left subtree will consist of the values between root and i while the right subtree will consist of the values between i (inclusive) and "n-1." Divide the pre[] that is given at index i and repeat for the left and right subtrees.\\
For example in {20, 16, 5, 18, 17, 19, 60, 85, 70}, 20 is the first element, so we make it root. Now we look for the first element greater than 20, we find 60. So we know the structure of BST is as following. \\
We recursively follow above steps for subarrays {5, 16, 17, 18, 19} and {60, 70, 85}, and get the complete tree. \\

For example, if the given traversal is {20, 16, 5, 18, 17, 19, 60, 85, 70}, then the output should be the root of the following tree.\\

\begin{forest}
for tree={
    grow=south,
    circle, draw, minimum size=3ex, inner sep=1pt,
    s sep=7mm
        }
[20
    [16
    [5]  [18 [17] [19]
    ]
    ]
    [60 [] [85 [70] []
    ]
    ]
]
\end{forest}\\

class Node\\
	int data;\\
	Node left, right;\\
	Node(int d)\\
		data = d;\\
		left = right = null;\\

class Index \\
	int index = 0;\\

class BinaryTree \\
	Index index = new Index();\\
	Node constructTreeUtil(int pre[], Index preIndex, int low, int high, int size)\\
		if (preIndex.index $>= size$ $||$ $low > high$)\\
			return null;\\
		Node root = new Node(pre[preIndex.index]);\\
		preIndex.index = preIndex.index + 1;\\
		if (low == high) \\
			return root;\\
		int i;\\
		for (i = low; $i <= high;$ ++i)\\ 
			if (pre[i] $>$ root.data)\\
				break;\\
		root.left = constructTreeUtil(pre, preIndex, preIndex.index, i - 1, size);\\
		root.right = constructTreeUtil(pre, preIndex, i, high, size);\\
		return root;\\
		
	Node constructTree(int pre[], int size)\\
		return constructTreeUtil(pre, index, 0, size - 1, size);\\
		
	void printInorder(Node node)\\
		if (node == null) \\
			return;\\
		printInorder(node.left);\\
		System.out.print(node.data + " ");\\
		printInorder(node.right);\\
		
	public static void main(String[] args)\\
		BinaryTree tree = new BinaryTree();\\
		int pre[] = new int[] { 20, 16, 5, 18, 17, 19, 60, 85, 70 };\\
		int size = pre.length;\\
		Node root = tree.constructTree(pre, size);\\
		System.out.println("Inorder traversal of the constructed tree is ");\\
		tree.printInorder(root);\\

Running time for this algorithm is O(N). we might need to perform O$(N^2)$ of these searches because our search tree might not be rooted at the BST's root. However, O(N) uses memorization in this case. Because there are only 3 possible states (min possible, max possible) for a given node, this is the case. The states are (-INF, +INF), (-INF, p.value), (g.value, p.value), where g is the most recent ancestor that we are the opposite subtree of from our parent. As an illustration, assuming the current node is a left child with parent p.\\

\begin{forest}
for tree={
    grow=south,
    circle, draw, minimum size=3ex, inner sep=1pt,
    s sep=7mm
        }
[10
    [5
    [1]  [7]
    ]
    [40 [] [50]
]
]
\end{forest}\\

Loop Invariant Proof for the BST preorder traversal and time complexity.\\
Initialization: Before the first iteration, our preorder array has the root left and right so lets say x is the root of the entire tree which is 10 in above tree. therefore if we traversal the tree our root is there and subtree rooted at node x.\\
Maintenance: Assume at the beginning of the procedure, it's true that if key k exists in the tree that it is in the subtree rooted at node x. There are three cases that can occur during the procedure:\\
case 1 : In first case if we got the root we traverse the tree from root x to left and then right till we get the leaves like first 10 then 5 then 1 after there is no childs back to root and right we have to complete the remaining.\\
case 2 : In these we check root and the next value if smaller then left side as these is property of BST we have to complete the tree.
case 3 : In this case we check root with next value which is greater than root then we have to travere the tree to right side node you can see the 40 node is root so value 50 is on right side.\\
Termination: By the loop invariant, we know that when the procedure terminates, as we traverse all the nodes which are elements from the array one by one at the last element we traverse the process ends here and the tree is traverse using BST.\\

T(n) = 2*T(n/2) + constant,\\
where constant is some constant (could be 1 or any other constant).\\
Using the Masters' Theorem,\\
we have T(n) = a*T(n/b) + f(n).\\
So, a=2, b=2, f(n) = constant since f(n) = $n^c$ = 1,\\
then it follows that c = 0 since f(n) is a constant.\\
From here, we can see that a = 2 and $b^c$ = 1.\\
So, $a>b^c$ or $2>2^0.$ So, $c < logb(a)$ or $0 < log2(2)$\\
From here we have T(n) = $\theta(n^{logb(a)}) = \theta(n^1) = \theta(n)$\\
So, $\theta(n)$ is both worst-case O(n), and best-case $\Omega(n)$ run-time.\\
T(n) = 2*T(n/2) + 1, where T(n) is the number of operations executed in your traversal algorithm (in-order, pre-order, or post-order makes no difference. hence it is prove it takes O(n) time complexity because we traverse all the nodes in the tree.\\

\\Reference:- Construct BST from given preorder traversal: Set 1, GeeksforGeeks, June 20, 2022, https://www.geeksforgeeks.org/construct-bst-from-given-preorder-traversa/

\item (25 Pts.) Recall that an inorder traversal of a binary search tree simply outputs all the keys in sorted order. Therefore, there is no unique way to reconstruct a binary search tree given its inorder traversal. In particular, no matter how unbalanced the tree originally was, one can always reconstruct a balanced one. Design an algorithm that, given a sorted array of n elements, constructs a balanced binary search tree out of them. Your algorithm should run in linear time. To get full credit you must show pseudo-code for your algorithm, show it is true using loop invariants, prove it’s running time with one of the three methods we discussed in class).
\item[Ans: ]    Is is similar to the above question but here we are going to traversal inorder where root is in between the left child and right child. so the sequesnce should be like this Left, Root, Right.\\ 

The Pseudo code for the Preorder Tree Traversal.\\
procedure inorder(p : pointer to a tree node)\\
    if p != nullptr\\
        inorder(p$->$left)\\
        Visit the node pointed to by p\\
        inorder(p$->$right)\\
    end if\\
end procedure\\

if we must create a BST from a sorted order. Any BST can be constructed. Therefore, a linear time complexity technique is presented here to determine the balanced BST. 
In a Binary Search Tree, the right child will typically be greater than the node and the left child will typically be less than that node for each sub-tree.\\ 
In the BST's inorder traversal, the left node, parent, and right child are explored in that order for each subtree.
This means, least element first, then the second least, and then the highest.
This means the inorder traversal of a BST gives the sorted order. 
From a sorted order of elements, there would be many possible Binary Search Trees possible. \\

Given Example for Inorder and preorder to construct BST.\\
Preorder: 1, 2, 4, 8, 9, 10, 11, 5, 3,  6, 7 (Root, Left, Right)\\
Inorder: 8, 4, 10, 9, 11, 2, 5, 1, 6, 3, 7 (Left, Root, Right)\\

In inorder to find the root is difficult because we dont where where is root because first is left then root and the right. So using help of preorder we can know the root and we will start constructing Inorder BST.\\

\begin{forest}
for tree={
    grow=south,
    circle, draw, minimum size=3ex, inner sep=1pt,
    s sep=7mm
        }
[1
    [2
    [4 [8]
    [9 [10] [11]
    ]
    ]
    [5]
    ]
    [3 [6] [7]
    ]
]
\end{forest}\\


 

Let's try to construct one, which is balanced.
The basic idea is to select the middle element and try to make it the root.
Recursively trying the same for every subtree.
For example:

Let the inorder array be: inorder {1, 2, 3, 4, 5}
The middle element is 3;
Root =3
Left of 3, are {1, 2}
Parent =2
Child=1 
Right of 3, are {4, 5}
Parent=4
Child=5

Since 3 is the root:\\

\begin{forest}
for tree={
    grow=south,
    circle, draw, minimum size=3ex, inner sep=1pt,
    s sep=7mm
        }
[3
    [1 [2]
    ]
    [4 [5]
    ]
]
\end{forest}\\

 

 

Pseudo-Code\\
Balanced BST from Inorder(inorder[], start, end)\\
            If $start > end$ \\
                        Return null\\
            Mid = (start+end)/2\\
            Allocate memory for the node "Root"\\
            Root$->$left = Balanced BST from Inorder(inorder,start,mid-1)\\
            Root$->$right=Balanced BST from Inorder(inorder,mid+1,end);\\
            Return Root\\
 
The time complexity of the above algorithm is O(n).\\
where I already show the time complexity proof using master theorem in Question 1.\\
In Master theorem I prove using the equation how it gives the O(n) time complexity. \\
T(n) = 2T(n/2)+ c\\
T(n/2) = 2T(n/4) + c => T(n) = 4T(n/4) + 2c + c\\
similarly T(n) = 8T(n/8) + 4c+ 2c + c\\
last step ... T(n) = nT(1) + c(sum of powers of 2 from 0 to h(height of tree))\\
so Complexity is $O(2^(h+1) -1)$\\
but h = log(n)\\
so, O(2n - 1) = O(n)\\
we can also show in Recursive tree method, In this method, a recursive tree is constructed. Then the time taken in each level is found and added to find the overall time complexity.\\

\\Reference:- Construct Tree from given Inorder and Preorder traversals, GeeksforGeeks, August 09, 2022, https://www.geeksforgeeks.org/construct-tree-from-given-inorder-and-preorder-traversal/
\\

\item (25 Pts.) Given an adjacency-list representation of a directed graph, how long does it take to compute the out-degree of every vertex? How long does it take to compute the in-degrees? Justify you answer.
\item[Ans: ]Adjacency-list representation of a directed graph:\\
Out-degree of each vertex\\
Graph out-degree of a vertex u is equal to the length of Adj[u].\\
The sum of the lengths of all the adjacency lists in Adj is $|E|$.\\
Thus the time to compute the out-degree of every vertex is $\theta$(V + E)\\

In-degree of each vertex\\
The in-degree of a vertex u is equal to the number of times it appears in all the lists in Adj.\\
If we search all the lists for each vertex, time to compute the in-degree of every vertex is $\theta$(VE)\\
Alternatively, we can allocate an array T of size $|V|$ and initialize its entries to zero.\\
We only need to scan the lists in Adj once, incrementing T[u] when we see 'u' in the lists.\\
The values in T will be the in-degrees of every vertex.\\
This can be done in $\theta$(V + E) time with Θ(V) additional storage.\\

The time taken to compute the out-degree of every vertex of a graph using a general adjacency list is O(V+E), where V is the number of vertices in the graph and E is the number of edges in the graph.\\
For a graph having n vertices say V1, V2, ... Vn, the adjacency list representation is as follows:
V1 = X, X, X\\
V2 = X, X, X\\
Vn = X, X, X\\

Where X ∈ {V1, V2,..., Vn}\\
X represents the nodes that have a direct edge from the vertex Vn.\\
The number of vertices that have a direct edge from a vertex V is known as its out-degree.\\ 
The list corresponding to that vertex must be visited in order to get the vertex's out-degree using the adjacency list. The out-degree of a vertex is determined by the quantity of nodes in the vertex's list. \\
The time it takes to traverse the list will be equal to the number of edges coming out of any given vertex, which may be retrieved directly using the adjacency list's index. \\
As a result, the amount of time required to traverse the lists of all the vertices will be equal to the graph's total number of edges, or E.\\
And the additional cost will be to access each vertex, that is V.
Therefore, the total cost to find the out-degree of every vertex will be O(V+E)\\

\\Reference:- 20.1 Representations of graphs, Algorithms Edition 4 Book. \\
Finding in and out degrees of all vertices in a graph, GeeksforGeeks, October 07, 2021, \\
https://www.geeksforgeeks.org/finding-in-and-out-degrees-of-all-vertices-in-a-graph/ \\
Proof for Distances from Tournament's Maximum Outdegree Vertex | Graph Theory, YouTube, July 25, 2021, https://www.youtube.com/watch?v=MyjmIGP0duY

\item (25 Pts.) Give an algorithm that determines whether or not a given undirected graph G = (V, E) contains a cycle. Your algorithm should run in O$(|V |)$, independent of number edges $|E|$. To get full credit you must show pseudo-code for your algorithm, show it is true using loop invariants, prove it’s running time with one of the three methods we discussed in class).

\item[Ans: ] To detect if there is any cycle in the undirected graph or not, we will use the DFS traversal for the given graph. For every visited vertex v, when we have found any adjacent vertex u, such that u is already visited, and u is not the parent of vertex v. Then one cycle is detected.\\

An undirected graph is acyclic (i.e., a forest) if a DFS yields no back edges.\\ Since back edges are those edges (u, v) connecting a vertex u to an ancestor v in a depth-first tree, so no back edges means there are only tree edges, so there is no cycle.\\ So we can simply run DFS. If find a back edge, there is a cycle.\\

The complexity is O(V) instead of O(E + V). Since if there is a back edge, it must be found before seeing $|V|$ distinct edges. This is because in a acyclic (undirected ) forest, $|E|$ $<=$ $|V |$ + 1\\

ALGORITHM:\\
   Check each edge one by one until at most $|V|$ edges, at which point we have a cycle and return.\\
INPUT:  Parent node, current node, visited nodes, adjacency matrix for g;\\
OUTPUT: Return no Cycle\\
Begin\\
Declare TotalEdges ( Parent node, current node, visited nodes, adjacency matrix for g)\\
For Each Node x In V\\
    For Each Edge y Of Node x\\
         IF IsNotMarked(y) Then\\
            Done\\
               TotalEdges++\\
               IF TotalEdges = $|V|$ RETURN "HAS CYCLE"\\
         End IF\\
    End For\\
End For\\
RETURN "NO CYCLE"\\
End\\

\\Program
class GFG \\
Boolean isCyclic(int V, LinkedList<Integer>[] alist) \\
    List<Integer> visited = new ArrayList<Integer>();\\
    for (int i = 0; i < V; i++) \\
        if (!visited.contains(i)) \\
            if (isCyclic(i, alist, visited, -1))\\
                return true;\\
    return false;\\

Boolean isCyclic(int vertex, LinkedList<Integer>[] alist, List<Integer> visited, int parent) \\
    visited.add(vertex);\\
    for (Iterator<Integer> iterator = alist[vertex].iterator();\\ iterator.hasNext();) \\
        int element = iterator.next();\\
        if (!visited.contains(element)) \\
            if (isCyclic(element, alist, visited, vertex))\\
                return true;\\
        else if (element != parent)\\
            return true;\\
    return false;\\

At most $|V| - 1$ edges make up a graph without cycles (a forest). Therefore, the DFS already found a cycle and terminates if it finds $|V|$ edges or more. As a result, the runtime is constrained by $O(|V|)$. \\
Consider first a graph with no cycles. Since there are O(V) edges, the graph is a forest, and the objective has been achieved. \\
Consider a graph with cycles, and your search algorithm will complete and report a successful search in the first cycle. Since the graph is undirected, there are only two alternatives when the algorithm inspects an edge: It has either traveled to the opposite end of the edge or it has and the circle is closed by this edge. There are only O(V) of these operations because once it notices the other edge vertex, it "inspects" it. The second scenario will only be encountered once while the algorithm is running.\\

\\Reference:- Detect cycle in an undirected graph, GeeksforGeeks, 23 June 2022, https://www.geeksforgeeks.org/detect-cycle-undirected-graph/\\

\end{enumerate}

\end{document}



